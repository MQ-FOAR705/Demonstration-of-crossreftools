\documentclass{article}
\usepackage[utf8]{inputenc}
\usepackage{lipsum}
\usepackage{cleveref}
%https://www.overleaf.com/learn/latex/Paragraph_formatting#Paragraph_spacing
\usepackage[utf8]{inputenc}
\usepackage[english]{babel}




\usepackage[colorlinks=true,linkcolor=blue]{hyperref}

\usepackage{hyperref}

% https://tex.stackexchange.com/a/418302/5482
\usepackage{crossreftools}

\newcommand{\includelabelintoc}{Error: }
% https://tex.stackexchange.com/a/436076/5483
\usepackage{tocloft}
\advance\cftsecnumwidth 0.5em\relax
\advance\cftsubsecindent 0.5em\relax
\advance\cftsubsecnumwidth 0.5em\relax




% Via Billy on the #latex channel:
% Is there a way to store template in the file so I can just press like, \error to bring up my error template?
% Because atm I always go back to the last error and copy-paste everything, then edit the lines out and replace them
% I had a look at using \newcommand but I can't see how I could input information into it once it compiles - I need something that pastes a whole bunch of code into the source code
% \newcommand{\error}{\subsubsection{\texorpdfstring{\underline{Error: ___}}{}}\label{error:er__}
% \begin{itemize}
%   \item
% \end{itemize}
% \begin{itemize}
% \renewcommand{\labelitemi}{}
% \item \underline{Solution:}
% \renewcommand{\labelitemi}{$\bullet$}
%   \item Intention:
%   Action:
%   Result:
% \end{itemize}}
% I suppose I could keep it handy as $comments, but I'd still have to manually delete the dollar sign from every line...

% https://en.wikibooks.org/wiki/LaTeX/Macros
% \newcommand{\name}[num]{definition}

% https://tex.stackexchange.com/a/36611/5483
% https://tex.stackexchange.com/a/123593/5483
\usepackage[explicit]{titlesec}
\usepackage{ulem}
\titleformat{\subsubsection}
  {\normalfont\large\itshape}{\thesubsubsection}{1em}{\uline{#1}} %make sure the subsub matches.




\newcommand{\itemizederror}[2]{%
\subsubsection{\includelabelintoc #1} % instead of hardcoding formatting in here, I adjust the subsubsection formatting in general. Obviously, if we want to signal different categories of subsection, we can move the global back into here.
% instead of hard-coding the error prompt, I pull it from above, so the change is consistent
\label{\includelabelintoc #1}
\begin{itemize} % I'm not putting \item here to support multiple bullets in the error definition.
    #2 
\end{itemize}
}


\newcommand{\solution}[3]{%
\subsubsection{Solution}
% https://en.wikibooks.org/wiki/LaTeX/List_Structures We want to use a descriptive list here.
\begin{description}
    \item [Intention:] #1
    \item [Action:] #2
    \item [Result:] #3
\end{description}
}

%\newcommand{\error}

\title{Demonstration of crossreftools}
\author{Brian Ballsun-Stanton}
\date{August 2019}

\begin{document}

\maketitle

\tableofcontents

% If we're using subsub sections, we need to redefine (AFTER we run the ToC command)
\advance\cftsecnumwidth 1em\relax


% so the list has a pretty name
\renewcommand
\listoflabelsname{List of Errors}

% Hack to do single-line skipping in list of errors
\setlength{\parskip}{-1em}

\crtlistoflabels


% We change parskip *after* create list of labels...

%This sets the indent length of a paragraph 
\setlength{\parindent}{0em}

%This sets the whitespace between paragraphs
\setlength{\parskip}{1em}

\section{Introduction}


\lipsum[1]
\subsection{Test subsection}
\lipsum[1]
\subsubsection{Test subsubsection}
\lipsum[1]
\subsection{Specific activity 1}
\lipsum[1]
\itemizederror{Test error via macro}{% 
\item \lipsum[1]
\item \lipsum[1]
}

\solution{\lipsum[1]}{\lipsum[1]}{\lipsum[1]}

\subsection{Error 1}
\label{Error: This was a silly error}
\lipsum[1]
\subsection{Specific activity 1}
\lipsum[1]
\subsection{Non-displayed Error 2}
\label{This was a different silly error that isn't displayed}
\lipsum[1]
\subsection{Displayed Error 3}
\label{Error: This was a different silly error that is displayed}
\lipsum[1-3]



\end{document}
